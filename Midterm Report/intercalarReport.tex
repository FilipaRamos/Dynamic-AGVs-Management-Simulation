%%%%%%%%%%%%%%%%%%%%%%%%%%%%%%%%%%%%%%%%%
% University Assignment Title Page 
% LaTeX Template
% Version 1.0 (27/12/12)
%
% This template has been downloaded from:
% http://www.LaTeXTemplates.com
%
% Original author:
% WikiBooks (http://en.wikibooks.org/wiki/LaTeX/Title_Creation)
%
% License:
% CC BY-NC-SA 3.0 (http://creativecommons.org/licenses/by-nc-sa/3.0/)
% 
% Instructions for using this template:
% This title page is capable of being compiled as is. This is not useful for 
% including it in another document. To do this, you have two options: 
%
% 1) Copy/paste everything between \begin{document} and \end{document} 
% starting at \begin{titlepage} and paste this into another LaTeX file where you 
% want your title page.
% OR
% 2) Remove everything outside the \begin{titlepage} and \end{titlepage} and 
% move this file to the same directory as the LaTeX file you wish to add it to. 
% Then add \input{./title_page_1.tex} to your LaTeX file where you want your
% title page.
%
%%%%%%%%%%%%%%%%%%%%%%%%%%%%%%%%%%%%%%%%%

%----------------------------------------------------------------------------------------
%	PACKAGES AND OTHER DOCUMENT CONFIGURATIONS
%----------------------------------------------------------------------------------------

\documentclass[12pt]{article}

\usepackage[utf8]{inputenc}
\usepackage[T1]{fontenc}
\usepackage{lmodern}
\usepackage{parselines}
\usepackage[portuguese]{babel}
\usepackage{graphicx}
\usepackage[document]{ragged2e}
\usepackage{listings}
\usepackage{xcolor}
\usepackage[margin=0.8in]{geometry}		
\usepackage{amsmath}
\usepackage{hyperref}
\usepackage{url}
\usepackage{float}
\usepackage{titlesec}
\usepackage{hyperref}
\usepackage{booktabs}
\usepackage{pgfplots}

\pgfplotsset{compat=1.5}
\graphicspath{ {images/} }

\titleclass{\subsubsubsection}{straight}[\subsection]

\newcounter{subsubsubsection}[subsubsection]
\renewcommand\thesubsubsubsection{\thesubsubsection.\arabic{subsubsubsection}}
\renewcommand\theparagraph{\thesubsubsubsection.\arabic{paragraph}} % optional; useful if paragraphs are to be numbered

\titleformat{\subsubsubsection}
  {\normalfont\normalsize\bfseries}{\thesubsubsubsection}{1em}{}
\titlespacing*{\subsubsubsection}
{0pt}{3.25ex plus 1ex minus .2ex}{1.5ex plus .2ex}

\makeatletter
\renewcommand\paragraph{\@startsection{paragraph}{5}{\z@}%
  {3.25ex \@plus1ex \@minus.2ex}%
  {-1em}%
  {\normalfont\normalsize\bfseries}}
\renewcommand\subparagraph{\@startsection{subparagraph}{6}{\parindent}%
  {3.25ex \@plus1ex \@minus .2ex}%
  {-1em}%
  {\normalfont\normalsize\bfseries}}
\def\toclevel@subsubsubsection{4}
\def\toclevel@paragraph{5}
\def\toclevel@paragraph{6}
\def\l@subsubsubsection{\@dottedtocline{4}{7em}{4em}}
\def\l@paragraph{\@dottedtocline{5}{10em}{5em}}
\def\l@subparagraph{\@dottedtocline{6}{14em}{6em}}
\makeatother

\setcounter{secnumdepth}{4}
\setcounter{tocdepth}{4}


\colorlet{punct}{red!60!black}
\definecolor{background}{HTML}{EEEEEE}
\definecolor{delim}{RGB}{20,105,176}
\colorlet{numb}{magenta!60!black}


\lstdefinelanguage{json}{
    basicstyle=\normalfont\ttfamily,
    numbers=left,
    numberstyle=\scriptsize,
    stepnumber=1,
    numbersep=8pt,
    showstringspaces=false,
    breaklines=true,
    frame=lines,
    backgroundcolor=\color{background},
    literate=
     *{0}{{{\color{numb}0}}}{1}
      {1}{{{\color{numb}1}}}{1}
      {2}{{{\color{numb}2}}}{1}
      {3}{{{\color{numb}3}}}{1}
      {4}{{{\color{numb}4}}}{1}
      {5}{{{\color{numb}5}}}{1}
      {6}{{{\color{numb}6}}}{1}
      {7}{{{\color{numb}7}}}{1}
      {8}{{{\color{numb}8}}}{1}
      {9}{{{\color{numb}9}}}{1}
      {:}{{{\color{punct}{:}}}}{1}
      {,}{{{\color{punct}{,}}}}{1}
      {\{}{{{\color{delim}{\{}}}}{1}
      {\}}{{{\color{delim}{\}}}}}{1}
      {[}{{{\color{delim}{[}}}}{1}
      {]}{{{\color{delim}{]}}}}{1},
}

\definecolor{dkgreen}{rgb}{0,0.6,0}
\definecolor{gray}{rgb}{0.5,0.5,0.5}
\definecolor{mauve}{rgb}{0.58,0,0.82}

\lstset{frame=tb,
  language=Java,
  aboveskip=3mm,
  belowskip=3mm,
  showstringspaces=false,
  columns=flexible,
  basicstyle={\small\ttfamily},
  numbers=none,
  numberstyle=\tiny\color{gray},
  keywordstyle=\color{blue},
  commentstyle=\color{dkgreen},
  stringstyle=\color{mauve},
  breaklines=true,
  breakatwhitespace=true,
  tabsize=3
}

\begin{document}

\begin{titlepage}

\newcommand{\HRule}{\rule{\linewidth}{1mm}} % Defines a new command for the horizontal lines, change thickness here

\center % Center everything on the page
 
%----------------------------------------------------------------------------------------
%	HEADING SECTIONS
%----------------------------------------------------------------------------------------

\includegraphics{feup.jpg}

\textsc{\large Agentes e Inteligência Artificial Distribuída}\\[0.8cm] % Major heading such as course name
\textsc{\large 4º ano do Mestrado Integrado em Engenharia Informática e Computação}\\[0.8cm] % Minor heading such as course title

%----------------------------------------------------------------------------------------
%	TITLE SECTION
%----------------------------------------------------------------------------------------

\HRule \\[1.2cm]
{ \huge \bfseries Simulação dinâmica da gestão de AGVs num contexto fabril}\\[0.6cm] % Title of your document
\HRule \\[3cm]
 
%----------------------------------------------------------------------------------------
%	AUTHOR SECTION
%----------------------------------------------------------------------------------------


% If you don't want a supervisor, uncomment the two lines below and remove the section above
\Large \emph{Authors:}\\[0.5cm] \normalsize
Daniel \textsc{Reis}\\[0.1cm] - up201308586 
- up201308586@fe.up.pt\\[0.1cm]
David \textsc{Baião}\\[0.1cm] - up201305195
- up201305195@fe.up.pt\\[0.1cm] 
Filipa \textsc{Ramos}\\[0.1cm] - up201305378
- up201305378@fe.up.pt\\[3cm] % Your name

%----------------------------------------------------------------------------------------
%	DATE SECTION
%----------------------------------------------------------------------------------------

{\large \today}\\[0cm] % Date, change the \today to a set date if you want to be precise

%----------------------------------------------------------------------------------------
%	TABLE OF CONTENTS & LISTS OF FIGURES AND TABLES
%----------------------------------------------------------------------------------------

\tableofcontents

%----------------------------------------------------------------------------------------
%	INTRODUÇÃO
%----------------------------------------------------------------------------------------

\section{Introdução} 

\justify\normalsize
No âmbito da unidade curricular de Agentes e Inteligência Artificial Distribuída pretende-se simular as operações de uma fábrica que utiliza AGVs para transporte de peças entre máquinas nas variadas fases de produção. Serão utilizadas plataformas de sistema e/ou frameworks na construção e simulação de agentes. O objectivo é manusear estas ferramentas por forma a auxiliar a construção de um sistema multiagente que permite comunicação e negociação entre agentes. Tem-se em vista a exploração das possibilidades destas plataformas atraves da construção de agentes com funções diferentes que se complementam uns aos outros e formam uma unidade de produção fabril.

Este projecto tem como objectivo principal a demonstração de possíveis aplicações de agentes de inteligência artificial no mundo prático. Unidades de produção como a que este projecto envisiona poderão ser uma realidade num futuro proximo e permitirão uma subida exponencial na eficiência e na organização de fabricas de produção em massa.

%----------------------------------------------------------------------------------------

\newpage % Start the article content on the second page, remove this if you have a longer abstract that goes onto the second page

%----------------------------------------------------------------------------------------
%	ESPECIFICAÇÃO
%----------------------------------------------------------------------------------------

\section{Especificação}

\subsection{Descrição do cenário}
Numa fábrica, existem variadas máquinas responsáveis pelas várias etapas da produção. Um lote é composto por um número variável de peças correspondendo, por exemplo, numa fábrica textil, a 10 camisolas. Um lote está pronto quando todas as suas peças passaram por todas as fases da linha de produção. A linha de produção é constituída por máquinas pertencentes a cada uma das fases específicas. Cada fase tem uma ou mais máquinas que fazem o mesmo trabalho. Sempre que uma máquina acaba de processar um lote negoceia com as máquinas responsáveis pela etapa seguinte a transferência do lote em causa. Após a determinação da máquina destino para a próxima etapa, a máquina de origem requere o transporte do lote a um AGV. Após ser feita uma negociação com todos os AGV's presentes no espaço, o escolhido transporta o lote para a máquina destino.

\begin{figure}[H]
  \centering
    \includegraphics[width=18cm, height = 10.5cm]{scenario.png}
  \caption{Cenário exemplificativo de uma fábrica simples.}
  \label{scenario}
\end{figure}

Como se pode observar no exemplo da figura \ref{scenario}, o processo de fabrico de um lote implica 3 fases distintas. Um lote entraria numa das máquinas da fase 1 e, quando o processo nesta fosse completado, a máquina que processou o lote negoceia com as máquinas da fase 2 para jogar com a disponibilidade das mesmas. Este processo repete-se até o lote ter sido processado numa máquina da última etapa. 

\newpage
Este cenário é constituído por:

\begin{description}
\item[Fase 1] 3 máquinas
\item[Fase 2] 2 máquinas 
\item[Fase 3] 3 máquinas
\end{description}

A negociação entre máquinas é feita na passagem da fase 1 para a fase 2 e da fase 2 para a fase 3. Os círculos vermelhos simbolizam AGV's em diferentes posições do espaço físico da fábrica. A negociação pelo transporte de um AGV implica a análise por parte do agente da distância do AGV à máquina origem e destino. O quadrado presente na parte inferior do esquema representa a estação de carregamento de um AGV. Este não é um agente. É apenas a localização referência para onde os AGV's se deslocam por forma a voltarem a encher a sua bateria.

Para efeitos de transporte por parte de um AGV será considerado que todas as peças de todos os lotes têm o mesmo peso (todos os lotes são compostos por peças iguais).

O espaço físico da unidade fabril está mapeada em coordenadas (\textit{xy}) que auxiliam à localização de todos os elementos do sistema.

\subsubsection{Máquinas}

No início, cada lote tem uma máquina da fase 1 aleatoriamente assignada de acordo com a capacidade de cada máquina. Para o caso da fábrica do exemplo da figura \ref{scenario}, as decisões a serem tomadas por cada máquina estão representadas na figura \ref{decisionMach}.

\begin{figure}[H]
  \raggedleft 
    \includegraphics[width=19cm, height = 6cm]{DecisionTreeMachines.png}
  \caption{Árvore de decisão para um lote a ser processado no exemplo \ref{scenario}.}
  \label{decisionMach}
\end{figure}

No processo de decisão sao tidas em conta as variáveis de cada máquina. Cada máquina é caracterizada por:

\begin{description}
\item[Capacidade de Processamento] -  Número inteiro de lotes que a máquina consegue processar ao mesmo tempo;
\item[Velocidade de Processamento] - Inteiro que avalia a velocidade com que a máquina processa um lote;
\item[Manutenção] - Booleano que indica se a máquina está operacional ou em manutenção;
\item[Localização] - Coordenadas da localização fixa da máquina.
\end{description}

Cada máquina pode continuar a receber lotes até que atinja a sua capacidade máxima de processamento. Todas estas variáveis são inerentes a uma máquina. Podem existir máquinas da mesma fase que não tenham as características iguais. A velocidade é a variável que auxilia ao cálculo do menor tempo em que os lotes são processados. Ao ser feita a escolha da máquina destino são tidos em conta tanto a capacidade de processamento como a velocidade de cada máquina por forma a obter a combinação que minimiza o tempo total de produção do lote.

\subsubsubsection{Grau de Potencial}

O \textbf{Grau de Potencial} de uma máquina expressa o equilíbrio entre as variáveis \textbf{capacidade de processamento} e \textbf{velocidade de processamento} e caracteriza-se pela fórmula seguinte:

\begin{equation}
G_{p}=\frac{1}{C_{p}}\times t_{p}
\label{eqG}
\end{equation}

em que $t_{p}$ representa: 

\begin{equation}
t_{p}=\frac{1}{V_{p}}
\label{eqT}
\end{equation}

\begin{table}[H]
\centering
\caption{Variáveis presentes nas equações \ref{eqG} e \ref{eqT}.}
\label{my-label}
\begin{tabular}{@{}p{2cm}ll@{}}
\toprule
\multicolumn{1}{c}{\textbf{Variável}} & \textbf{Definição}   & \multicolumn{1}{c}{\textbf{Unidades}} \\ \midrule
\multicolumn{1}{c}{$G_{p}$} & Grau de potencial &  \multicolumn{1}{c}{-}  \\ \midrule
\multicolumn{1}{c}{$C_{p}$} & Capacidade de processamento &  \multicolumn{1}{c}{-}  \\ \midrule
\multicolumn{1}{c}{$t_{p}$} & Tempo de processamento      & \multicolumn{1}{c}{$s$} \\ \midrule
\multicolumn{1}{c}{$V_{p}$} & Velocidade de processamento  & \multicolumn{1}{c}{$lotes/s$} \\ \bottomrule
\end{tabular}
\end{table}

Quanto menor o grau de potencial melhor é a opção. Assim, uma máquina escolhe sempre a máquina da fase seguinte com menor grau de potencial. Um menor grau de potencial significa menor tempo de processamento de um lote e este é o objectivo da unidade fabril. Desta forma, o grau de potencial é inversamente proporcional à capacidade de processamento (observável em baixo) e diretamente proporcional ao tempo de processamento (declive da reta é igual a $C_{p}^{-1}$, observável em baixo).

\begin{tikzpicture}
\centering
  \begin{axis}[ 
    title=Variação do grau de potencial em relação à capacidade de processamento,
    xlabel=$C_{p}$,
    ylabel={$G_{p}$},
  ] 
    \addplot[draw=blue][domain=0:5]{1/x};
  \end{axis}\end{tikzpicture}

\begin{tikzpicture}
\centering
  \begin{axis}[ 
    title=Variação do grau de potencial em relação ao tempo de processamento,
    xlabel=$t_{p}$,
    ylabel={$G_{p}$},
  ] 
    \addplot[draw=blue][domain=0:5]{x};
  \end{axis}\end{tikzpicture}

\subsubsection{AGV}

Os AGV's (\textit{Automated Guided Vehicles}) vão estar espalhados pelo espaço físico da loja, navegando pelo sistema para os sítios onde são necessários. Efetuam negociações diretas com as máquinas que requerem transporte. Todos os AGV's conhecem as localizações de todas as máquinas desde o início da simulação. 

As variáveis que cada AGV tem de ter em conta são as expressadas a seguir:

\begin{description}
\item[Capacidade de carga] - Número de lotes que um AGV consegue carregar;
\item[Autonomia] - Quantidade de bateria ainda disponível;
\item[Localização] - Coordenadas da localização do AGV em cada momento.
\end{description}

\subsection{Objectivos}
\justify\normalsize
O maior objectivo inerente a esta simulação é o de verificar a possibilidade de automação de uma unidade fabril por forma a diminuir a interação humana o mínimo possivel e visando obter a maior eficiência possivel na produção. Do ponto de vista da aplicação de agentes o ponto mais evidente é a observação da comunicação e do comportamento de e entre agentes máquina e AGV. 

Será assumido na simulação que é do interesse da fábrica obter a maior eficiência possível na sua linha de produção. Isto significa que todas as escolhas de agentes terão em vista obter 


\subsection{Resultados esperados}
\justify\normalsize
É suposto o programa ficar funcional na data de entrega, logo o resultado esperado é isto estar pronto.

\subsection{Avaliação dos resultados}
\justify\normalsize
20/20


%----------------------------------------------------------------------------------------
%	JADE, REPAST+SaJas
%----------------------------------------------------------------------------------------

\section{Ferramentas}

\subsection{Explicação das plataformas a usar}
JADE e SAJAS

\subsubsection{Características principais}


\subsubsection{Funcionalidades Relevantes}


%----------------------------------------------------------------------------------------
%	Agentes
%----------------------------------------------------------------------------------------

\section{Sistema}

\subsection{Identificação e caracterização dos agentes}
Máquinas e AGV's. As máquinas processam lotes e os AGV tratam do transporte destes. Quando uma máquina acaba o processamento, deverá negociar com uma nova máquina o transporte da peça que deverá ser feito pelos AGV.

\subsection{Interação entre agentes}
As máquinas processam os lotes/peças e comunicam com as outras máquinas a próxima fase. Após decidirem qual deverá ser a próxima máquina a processar as peças devem comunicar com os AGV disponíveis o transporte entre a máquina atual e a seguinte no processo.

\subsection{Faseamento}
\textbf{FASEAMENTO NÃO É ISTO! É AS FASES DE DESENVOLVIMENTO DO PROJETO!}


Fase 1- Processamento de lote/peças por uma máquina (máquina de origem).
\newline
Fase 2- Determinação da máquina responsável pela etapa seguinte.
\newline
Fase 3- Negociamento de transporte entre a máquina de origem e a máquina responsável pela etapa seguinte.
\newline
Fase 4- Comunicação entre máquinas e AGV para efetuar transporte entre máquina de origem e máquina seguinte.
\newline
Fase 5- Transporte efetuado pelas AGV entre máquinas.
\newline
Fase 6- Jump to Fase 1.



\section{Conclusões}
\justify\normalsize


\bibliography{title_page_1}
\bibliographystyle{plain}

\end{titlepage}
\end{document}